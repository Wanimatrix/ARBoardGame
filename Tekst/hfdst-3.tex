\chapter{Legoblokdetectie op basis van CAD modellen}
\label{hoofdstuk:3}
In deze methode worden legoblokken gedetecteerd door alle geometrische informatie over de legoblok te gebruiken. Dit kan door vanuit enorm veel zichtpunten een 2D afbeelding te genereren van een CAD model van een legoblok. In deze afbeeldingen kunnen vervolgens features worden berekend waarmee dan, via matching op elke frame, de legoblok in de frame kan worden gevonden~\cite{aubry2014seeing}. Een groot voordeel aan deze techniek is dat het steeds gebruik maakt van alle geometrische informatie over de legoblok, hierdoor kunnen legoblokken gevonden worden in een frame in om het even welke pose. Het grootste nadeel is de performantie want, aangezien we een grote hoeveelheid aan zichtpunten nodig hebben om de pose te kunnen inschatten, moeten enorm veel features met elkaar gematcht worden. Daarom is ook de keuze van het soort features erg belangrijk.

In dit hoofdstuk worden verschillende types van features met elkaar vergeleken.

TODO: BESPREEK KORT RESULTATEN VAN DIT HOOFDSTUK %TODO

\section{Features}
\subsection{Haar-like}
\subsection{LBP}
\subsection{HOG}
\subsection{Parts-based}

\section{Vergelijking}

Om al deze type features met elkaar te vergelijken werden


~\cite{viola2001rapid}

TODO: BESCHRIJVING METHODES: CCLA: HAAR; LBP; HOG, SVM: HOG; parts-based HOG %TODO

%\subsection{Een item}
%De bijbehorende tekst. Denk eraan om de paragrafen lang genoeg te maken en
%de zinnen niet te lang.
%
%Een paragraaf omvat een gedachtengang en bevat dus steeds een paar zinnen.
%Een paragraaf die maar \'e\'en lijn lang is, is dus uit den boze.

\section{Hoe goed werken ze?}
%Er zijn in een hoofdstuk verschillende onderwerpen. We zullen nu
%veronderstellen dat dit het laatste onderwerp is.

%\subsection{Een item}
%Maak ook geen misbruik van opsommingen. Voor korte opsommingen gebruik je
%geen ``\verb|itemize|'' of ``\texttt{enumerate}'' commando's. Doe dus
%\emph{niet} het volgende:
%\begin{quote}
%  De Eiffeltoren bevat drie verdiepingen:
%  \begin{itemize}
%  \item de eerste;
%  \item de tweede;
%  \item de derde.
%  \end{itemize}
%\end{quote}
%Maar doe:
%\begin{quote}
%  De Eiffeltoren bevat drie verdiepingen: de eerste, de tweede en de derde.
%\end{quote}

\section{Besluit van dit hoofdstuk}
BESLUIT VAN DIT HOOFDSTUK %TODO 
%Als je in dit hoofdstuk tot belangrijke resultaten of besluiten gekomen
%bent, dan is het ook logisch om het hoofdstuk af te ronden met een
%overzicht ervan. Voor hoofdstukken zoals de inleiding en het
%literatuuroverzicht is dit niet strikt nodig.

%%% Local Variables: 
%%% mode: latex
%%% TeX-master: "masterproef"
%%% End: 
